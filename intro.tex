\section{Введение}

В проектах «Компилятор формул» и «Интерпретатор арифметических
выражений» были решены задачи расширения языков стекового калькулятора
и интерпретатора арифметических выражений операциями побитового сдвига.
Применена структура данных <<хэш-таблица>>.Решение задачи требовало представления
о формальных грамматиках, основы ООП и знания языка Ruby.

Проект «Выпуклая оболочка»\cite{convex} решает задачу индуктивного перевычисления выпуклой оболочки последовательно поступающих точек плоскости и таких её характеристик, как периметр и площадь. Целью данной работы является
индуктивное вычисление расстояние до наперёд заданного стандартного прямоугольника
и количества острых улов выпуклой оболочки. Применено два специфических алгоритма,
значительно ускоряющих вычисление расстояние от прямоугольника до
отрезка~\cite{seginters} и проверку пересечения прямоугольника и
отрезка (алгоритм Лианга-Барски)~\cite{barsky}. Решение этой задачи требует знания
теории индуктивных функций, основ аналитической геометрии и векторной алгебры
и языка Ruby~\cite{ruby}.

Проект «Изображение проекции полиэдра»~\cite{polyedr}~--- пример
классической задачи, для успешного решения которой необходимо знакомство с
основами вычислительной геометрии. Задачей, решаемой в данной работе, является
модификация эталонного проекта с целью определения суммы длин полностью видимых
рёбер заданного полиэдра. Для этого необходимы хорошее понимание ряда разделов
аналитической геометрии и векторной алгебры, основ объектно-ориентированного
программирования и языка Ruby.

Для подготовки пояснительной записки необходимо знакомство с программой компьютерной вёрстки \LaTeX~\cite{rlatex}, умение набирать математические формулы~\cite{texbook} и включать в документ графические изображения и исходные
коды программ.

Общее количество строк в рассмотренных проектах составляет около $190$, из которых $28$ были изменены или добавлены автором в процессе работы над задачами модификации.
