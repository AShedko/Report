\section{Модификация проекта «Выпуклая оболочка»}

\subsection{Постановка задачи}
Требуется модифицировать эталонный проект:
\begin{enumerate}
\item для индуктивного вычисления количества всех острых внутренних углов выпуклой оболочки~(задача 40).
\item для индуктивного вычисления расстояния от выпуклой оболочки до заданного стандартного прямоугольника~(Задача 52).
\end{enumerate}
\subsection{Теоретические аспекты}

Задача построения выпуклой оболочки множества точек может быть сформулирована следующим 
образом: для множества точек $M$ необходимо найти наименьшее \emph{выпуклое} множество, 
включающее $M$. \emph{Выпуклым} будем называть любое множество $M$, удовлетворяющее условию
$ \forall x_1, x_2 \in M\ [x_1,x_2]\in M.$

Пусть $X$~---множество точек на плоскости $\mathbb{R}^2$, $\mathcal{P}$~--- множество всех выпуклых фигур на плоскости. Тогда тройка $(f,g,h)$, где 
$f\colon X^* \rightarrow \mathcal{P}$~--- \emph{выпуклая оболочка последовательности точек}, $g\colon X^* \rightarrow \mathbb{N}$~--- \emph{количество острых углов в ней}, $h\colon X^* \rightarrow \mathbb{R}$~--- \emph{расстояние от неё до стандартного прямоугольника}, задаёт индуктивную функцию $$F\colon X^* \rightarrow \mathcal{P} \times \mathbb{N} \times \mathbb{R}, F = \begin{pmatrix}f\\ g\\ h\end{pmatrix}.$$