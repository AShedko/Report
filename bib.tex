\begin{thebibliography}{}

\bibitem{convex}
\link{https://home.mephi.ru/files/2373/material\_ici\_toc.zip/index.html}~---
Описание проекта «Выпуклая оболочка».

\bibitem{seginters}
{\em Advances in Spatial and Temporal Databases: 9th International Symposium.}~---
SSTD 2005, Angra Dos Reis Brazil, August 22-24, 2005, Proceedings, С 333.

\bibitem{barsky}
Liang, Y.D., and Barsky, B., {\em A New Concept and Method for Line Clipping.}~---
ACM Transactions on Graphics, 3(1):1-22, January 1984.

\bibitem{compf}
\link{https://home.mephi.ru/files/2077/material\_ici\_toc.zip/index.html}~---
Описание проекта \emph{<<Стековый компилятор формул>>}
Е.А. Роганов

\bibitem{rubydoc}
\link{http://ruby-doc.org/}~---
Документация языка Ruby

\bibitem{ruby}
\link{http://ru.wikipedia.org/wiki/Ruby}~---
Википедия (свободная энциклопедия) о языке Ruby.

\bibitem{polyedr}
\link{???}~---
Описание проекта «Изображение проекции полиэдра».

\bibitem{rlatex}
С.М. Львовский.
{\em Набор и вёрстка в системе \LaTeX, 3-е изд., испр. и доп.}~---
М., МЦНМО, 2003. Доступны исходные тексты этой книги.

\bibitem{texbook}
D.~E.~Knuth. {\em The \TeX{}book.}~---
Addison-Wesley, 1984. Русский перевод:
Дональд~Е.~Кнут.
{\em Все про \TeX.}~--- Протвино, РД\TeX, 1993.

\end{thebibliography}
