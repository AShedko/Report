\section{Модификация проекта «Интерпретатор арифметических выражений»}

\subsection{Постановка задачи:}
Вычисляются значения выражений, содержащих битовые операциями  $<<$ и $>>$,
приоритет первой из которых является минимальным, а второй — максимальным.

\subsection{Теоретические аспекты:}
Для решения задачи требуется учесть приоритет оператора сдвига вправо, а потому
граматика $G_0$ из предыдущего пункта должна быть изменена:

$G_0$:
\medskip
\noindent\hspace{2cm}
\begin{tabular}{rllllll}
$S_0 \rightarrow$ & $F $&$\mid$&$ S~L~F$\\
$F \rightarrow$ & $T $&$\mid$&$ F+T$&$\mid$&$ F-T$\\
$T \rightarrow$ & $M $&$\mid$&$ T*M$&$\mid$&$ T~/~M$\\
$S_1 \rightarrow$ & $M $&$\mid$&$ S~R~F$\\
$M \rightarrow$ & $(S_0)$&$\mid$&$ V$\\
$V \rightarrow$ & a $\mid$ b $\mid$c $\mid$ \dots $\mid$ z\\
\end{tabular}
\medskip

$S_0$ --- Класс сдивгаемых влево выражений;
$S_1$ --- Класс сдивгаемых вправо выражений

$G_s$ можно оставить без изменений.
