\section{Модификация проекта «Компилятор формул»}

\subsection{Постановка задачи:}
\begin{enumerate}
\item Модификация калькулятора:
В предположении, что язык стекового калькулятора
расширен операциями L (left) и R (right),
реализующими побитовый сдвиг влево и вправо соответственно, компилировать
формулы, содержащие операции $<<$ и $>>$.
\item Модификация интерпретатора:
Вычисляются значения выражений, содержащих битовые операциями << и >>,
приоритет первой из которых является минимальным, а второй — максимальным.
\end{enumerate}

Здесь следует описать:
\begin{enumerate}[1)]
\item точную постановку задачи;
\item изложение необходимых для решения задачи теоретических аспектов;
\item описание используемых структур данных и применяемых алгоритмов;
\item возможные обобщения рассматриваемой задачи (не обязательно, но
      весьма желательно).
\end{enumerate}
