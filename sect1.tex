\section{Модификация проекта «Компилятор формул»}

\subsection{Постановка задачи:}
\begin{enumerate}
\item Модификация калькулятора:
В предположении, что язык стекового калькулятора
расширен операциями L (left) и R (right),
реализующими побитовый сдвиг влево и вправо соответственно, компилировать
формулы, содержащие операции $<<$ и $>>$.
\item Модификация интерпретатора:
Вычисляются значения выражений, содержащих битовые операциями  $<<$ и $>>$,
приоритет первой из которых является минимальным, а второй — максимальным.
\end{enumerate}

\subsection{Теоретические аспекты:}
С формальной точки зрения компилятор представляет собой программную реализацию
некоторой функции $\tau\colon L_1 \rightarrow L_2$, действующей из множества
цепочек одного языка $L_1$ в множество цепочек другого $L_2$ таким образом, что
$\forall \omega \in L_1$ {\em семантика цепочек $\omega$ и $\tau(\omega)\in L_2$
 совпадает.}

Для решения этих задач необходимо задать грамматики, описывающие языки стекового
калькулятора и компилятора, соответственно $G_0$ и $G_s$:

$G_0$:
\medskip
\noindent\hspace{2cm}
\begin{tabular}{rll}
$S \rightarrow$ & $F \mid S~R~F\mid S~L~F$\\
$F \rightarrow$ & $T \mid F+T \mid F-T\\
$T \rightarrow$ & $M \mid T*M \mid T~/~M$\\
$M \rightarrow$ & $(F) \mid V$\\
$V \rightarrow$ & $ a \mid b \mid c \mid \dots \mid z$
\end{tabular}
\medskip

$G_s$:
\medskip
\noindent\hspace{2cm}
\begin{tabular}{rll}
$e\rightarrow$ & $e~e~+ \mid e~e~-\mid e~e~* \mid e~e~/ \mid e~e~>>\mid e~e~<<\mid$\\
&$\mid a \mid b \mid \dots \mid z$
\end{tabular}

 где $L$, $R$ соответсвуют $<<$, $>>$, т.е. сдвигу влево и вправо, а $S$ -- классу

Здесь следует описать:
\begin{enumerate}[1)]
\item точную постановку задачи;
\item изложение необходимых для решения задачи теоретических аспектов;
\item описание используемых структур данных и применяемых алгоритмов;
\item возможные обобщения рассматриваемой задачи (не обязательно, но
      весьма желательно).
\end{enumerate}
