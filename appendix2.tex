\section{Приложение Б}

\begin{enumerate}

\item
Так как используемый по умолчанию для просмотра pdf-файлов Adobe Reader не 
поддерживает возможности перезагрузки модифицированной версии документа, то
необходимо создать в своей домашней директории файл \verb|.latexmkrc|
со следующей строкой (без начальных пробелов):

\begin{verbatim}
  $pdf_previewer= "start evince";
\end{verbatim} 

\item
Для получения и просмотра итогового документа необходимо выполнить команду

\begin{verbatim}
 latexmk -pdf -pvc paper.tex
\end{verbatim} 

\noindent
Скрипт \verb|latexmk| анализирует время модификации всех файлов, от которых
зависит итоговый документ \verb|paper.pdf|; выполняет при необходимости
нужное количество раз команду \verb|pdflatex paper.tex|; запускает программу,
позволяющую увидеть итоговый документ, или посылает этой программе
сигнал о необходимости перезагрузки модифицированной версии документа. 

\item
Обязательно следует соблюдать рекомендации по набору русского языка, 
оформлению рисунков, таблиц, текстов алгоритмов и программ, сформулированные
в шаблоне пояснительной записки.

\item
Объём пояснительной записки к курсовой работе должен быть не менее
20 страниц (включая титульный лист, аннотацию, содержание, введение,
основные разделы, список литературы и интернет-ресурсов, приложения).

\item
Пояснительная записка не должна содержать сколь-либо объёмных фрагментов
текста, заимствованных из каких-либо источников. 

\item
Настоятельно рекомендуется следить за сообщениями об ошибках и 
предупреждениях, появляющихся в файле \verb|paper.log|, и исправлять их.

\item
Если Вы захотите установить/использовать \TeX на своём домашнем компьютере,
то ознакомьтесь с рекомендациями, размещёнными в сети Интернет по адресу
\link{http://www.tug.org/texlive/doc/texlive-ru/texlive-ru.html}.

\end{enumerate}
